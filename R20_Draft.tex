\documentclass{article}
\usepackage[utf8]{inputenc}
 
\title{Sections and Chapters}
\author{}
\date{\today}
 
\begin{document}
 
\maketitle
 
\section{Introduction}

\section{Base Elements of a Character}
General Combat
speed 
accuracy
dodge
balance
knock-back
stun

 
Melee combat
parry
counter

Ability Scores
---------------
ST – damage w/ melee weapons, rate/frequncy of melee attack
Dex – hand eye coordination, ranged combat critical hit rate (ranged and melee) , accuracy to hit
Con – physical health, heartyness, hp
Agility – grace, balance, ac, dodge
int – damage w/ magic through masteries e.g. maximize, enlarge, extend, and dely spell. Determines combat advantage
mnd – accuracy of spells (debufs, enfeebs, attacks) magical defence (magic ac)

HPs
------------
What do they represent

Concept of Health vs HPs
------------
Combat and damage endurance
Health, life, and lethal damage

Skills
------------
\section{Event-based skill progression}
* Weapon Skills
Weapon skills increase based on use.
By successfully connecting with a challenging target skill is rewarded.
The more challenging the target, the greater the rate of increased skill reward.

Unarmed
Light Melee
One-handed melee
two-handed melee
ranged
light melee martial
one-handed melee martial
(more weapon skills)


\section{Allocated skill-point progression}



\section{Compound Elements}
This expounds on section 'Base Elements of a Character'.
In this section we further describe elements of a character, but the elements are composites of the base elements that we use to calculate them. 

How to calculate an attack
----------------------------






 
\section{Gameplay Philosophies}
RPG characters do two things, engage in social situations and engage in combat. Social engagements are largely entangled with roll playing and by the choices made by the player and kept in check by the GM. These actions extend to talking to others PCs and NPCs haggling, decision making, leadership (or lack thereof), creativity, planning, diplomacy, and charm. Skill based checks are used in the game to keep interactions in check by supplementing or taxing the efforts of the players. These can only go so far, as it would take away from the game to allow rolls for the player to make better choices. However, the player may not be an elegant public speaker, but their character may be. For situations akin to the latter we roll to supplement the players ability. In the converse situation, rolls are made to determine outcomes and may penalize an otherwise impressive performance from a player if the character is ill-equipped for a situation. Because of this difficulty no ability stat should influence social interactions, such as a case with charisma being a stat. This can lead charisma being used as a dump stat and players relying on their own personal skills in place of roll playing the character they have. It would be just as absurd to allow a disengaged player to make high charisma-based rolls in place of dialog or any social investment. Because of these reasons, social rolls are to be kept to a minimum as a mechanic of the game and used only to enhance the immersiveness of the game.  


\section{Design Decision Notes}
 
Base Attack Bonus: This is removed. The concept of BAB is predicated on a class-based system where more martial classes are granted greater martial prowess. In this game there are no classes. Players choose what there characters do and if they use a weapon often they become more proficient with that weapon reguardless if they spend most of there time casting spells.
 
 
 
\end{document}

